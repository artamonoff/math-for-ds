% !TEX root = exercises-math-for-ds.tex

\subsection{Операции с матрицами}

\subsubsection{Скалярное умножение и сложение}

\begin{exercise}
Рассмотрим матрицы
\begin{align*}
	A&=\begin{pmatrix}
		-1 & 2 & 0 \\ 0 & 2 & 3 \\ 1 & -1 & 0 \\ 2 & -2 & 0
	\end{pmatrix} &
	B&=\begin{pmatrix}
		0 & -1 & 1 \\ 0 & 1 & 1 \\ 2 & 0 & -1 \\ 1 & 1 & 2
	\end{pmatrix} &
	C&=\begin{pmatrix}
		2 & 0 & -1 \\ -1 & -2 & 2 \\ 1 & -1& 2 \\ 0 & 3 & -1
	\end{pmatrix}
\end{align*}
Вычислите
\begin{align*}
	& 2A+B & &A-2C & &4B-A-C & &C-2A+4B
\end{align*}
\end{exercise}

\begin{exercise}
Рассмотрим матрицы
\begin{align*}
	A&=\begin{pmatrix}
		2 & 1 & 5 \\ 3 & 4 & 3 \\ 1 & 2 & 0 \\ 2 & 3 & 1 \\ 1 & 1 & 0
	\end{pmatrix} &
	B&=\begin{pmatrix}
		2 & 1 & 0 \\ 2 & 5 & 2 \\ 4 & 3 & 2 \\ 3 & 4 & 1 \\ 1 & 3 & 2
	\end{pmatrix} &
	C&=\begin{pmatrix}
		5 & 2 & 3 \\ 2 & 3 & 0 \\ 2 & 1 & 0 \\ 1 & 0 & 1 \\ 2 & 2 & 3
	\end{pmatrix}
\end{align*}
Вычислите
\begin{align*}
	& A+3B & &3B-2C & &2B-C+3A & &2C+3A-5B
\end{align*}
\end{exercise}

\begin{exercise}
Рассмотрим матрицы
\begin{align*}
	A&=\begin{pmatrix}
		-1 & 2 & 2 & 1 & 0 \\ 1 & 0 & -2 & 1 & 0
	\end{pmatrix} \\
	B&=\begin{pmatrix}
		0 & 0 & 1 & 3 & 2 \\ -1 & 0 & 2 & 1 & -3
	\end{pmatrix} \\
	C&=\begin{pmatrix}
		1 & 2 & 0 & 1 & 0 \\ 0 & 1 & 1 & -1 & -2
	\end{pmatrix}
\end{align*}
Вычислите
\begin{align*}
	& 3A-B & &2A-C & &2B-C+3A & &B-2A+C
\end{align*}
\end{exercise}

\subsubsection{Умножение метриц}

\textbf{Замечание}: через \(\odot\) будем обозначать \textit{произведение Адамара} для матриц

\begin{exercise}
Для следующим матриц вычислите \(A\odot B\), если операция определена
\begin{enumerate}
	\item \(A=\begin{pmatrix} 1 & 2 \\ 0 & -1 \end{pmatrix}\) 
	\(B=\begin{pmatrix} -1 & 1 \\ 2 & -2 \end{pmatrix}\)
	\item \(A=\begin{pmatrix} 1 & 1 & 0 \\ 0 & 1 & 1 \\ 1 & 0 & 1 \end{pmatrix}\) 
	\(B=\begin{pmatrix} -1 & 0 & 1 \\ 1 & -1 & 0 \\ 0 & 1 & -1 \end{pmatrix}\)
	\item \(A=\begin{pmatrix} 1 & 0 & -1 & 1\\ 1 & 2 & -1 & 0  \end{pmatrix}\) 
	\(B=\begin{pmatrix} -1 & 1 & 1 & 2 \\ 0 & 1 & 2 & -2 \end{pmatrix}\)
	\item \(A=\begin{pmatrix} 1 \\ -1 \\ 2 \\ 0 \\ -2  \end{pmatrix}\) 
	\(B=\begin{pmatrix} 0 \\ 2 \\ -1 \\ 1 \\ 0  \end{pmatrix}\)
\end{enumerate}
\end{exercise}

\begin{exercise}
Для матрицы \(A=\begin{pmatrix} 1 & 1 \\ 0 & 1 \end{pmatrix}\) вычислите
\begin{align*}
	&A\odot A & &A^\top\odot A & &A\odot A\odot A & &A\odot A^\top\odot A & &A\odot A^\top\odot A^\top
\end{align*}
\end{exercise}

\begin{exercise}
Для матрицы \(A=\begin{pmatrix} 1 & 1 & 0 \\ 0 & 1 & 1 \\ 1 & 1 & 0 \end{pmatrix}\) вычислите
\begin{align*}
	&A\odot A & &A^\top\odot A & &A\odot A\odot A & &A\odot A^\top\odot A & &A\odot A^\top\odot A^\top
\end{align*}
\end{exercise}

\begin{exercise}
Рассмотрим матрицы
\begin{align*}
	A&=\begin{pmatrix}
		-1 & 2 & 0 \\ 0 & 2 & 3 \\ 1 & -1 & 0 \\ 2 & -2 & 0
	\end{pmatrix} &
	B&=\begin{pmatrix}
		0 & -1 & 1 \\ 0 & 1 & 1 \\ 2 & 0 & -1 \\ 1 & 1 & 2
	\end{pmatrix} &
	C&=\begin{pmatrix}
		2 & 0 & -1 \\ -1 & -2 & 2 \\ 1 & -1& 2 \\ 0 & 3 & -1
	\end{pmatrix}
\end{align*}
Вычислите
\begin{align*}
	& A\odot B\odot C & &A\odot B-C & &2B\odot C-A& &2A\odot B-3B\odot C
\end{align*}
\end{exercise}

\begin{exercise}
Рассмотрим матрицы
\begin{align*}
	A&=\begin{pmatrix}
		2 & 1 & 5 \\ 3 & 4 & 3 \\ 1 & 2 & 0 \\ 2 & 3 & 1 \\ 1 & 1 & 0
	\end{pmatrix} &
	B&=\begin{pmatrix}
		2 & 1 & 0 \\ 2 & 5 & 2 \\ 4 & 3 & 2 \\ 3 & 4 & 1 \\ 1 & 3 & 2
	\end{pmatrix} &
		C&=\begin{pmatrix}
		5 & 2 & 3 \\ 2 & 3 & 0 \\ 2 & 1 & 0 \\ 1 & 0 & 1 \\ 2 & 2 & 3
	\end{pmatrix}
\end{align*}
Вычислите
\begin{align*}
	& A\odot B\odot C & &2A\odot B-C & &B\odot C+2A& &3A\odot B-2B\odot C
\end{align*}
\end{exercise}
	
\begin{exercise}
Рассмотрим матрицы
\begin{align*}
	A&=\begin{pmatrix}
		-1 & 2 & 2 & 1 & 0 \\ 1 & 0 & -2 & 1 & 0
	\end{pmatrix} \\
	B&=\begin{pmatrix}
		0 & 0 & 1 & 3 & 2 \\ -1 & 0 & 2 & 1 & -3
	\end{pmatrix} \\
	C&=\begin{pmatrix}
		1 & 2 & 0 & 1 & 0 \\ 0 & 1 & 1 & -1 & -2
	\end{pmatrix}
\end{align*}
Вычислите
\begin{align*}
	& & A\odot B\odot C & &2A\odot C-B & &B\odot C-2B& &3A\odot C-2A\odot C
\end{align*}
\end{exercise}

\begin{exercise}
Для следующим матриц вычислите произведении \(AB\) и \(BA\), если операции определены
\begin{enumerate}
	\item \(A=\begin{pmatrix} 1 & 2 \\ 0 & -1 \end{pmatrix}\) 
	\(B=\begin{pmatrix} -1 & 1 \\ 2 & -2 \end{pmatrix}\)
	\item \(A=\begin{pmatrix} 1 & 2 \\ 0 & -1 \end{pmatrix}\) 
	\(B=\begin{pmatrix} 1 & 1 & -1 \\ 0 & 2 & 1 \end{pmatrix}\)
	\item \(A=\begin{pmatrix} 1 & 1 & 0 \\ 0 & 1 & 1 \\ 1 & 0 & 1 \end{pmatrix}\) 
	\(B=\begin{pmatrix} -1 & 0 & 1 \\ 1 & -1 & 0 \\ 0 & 1 & -1 \end{pmatrix}\)
	\item \(A=\begin{pmatrix} 1 & 0 & -1 & 1\\ 1 & 1 & -1 & 0  \end{pmatrix}\) 
	\(B=\begin{pmatrix} -1 & 1 \\ 1 & -1 \\ 0 & 1 \\ 1 & -1 \end{pmatrix}\)
	\item \(A=\begin{pmatrix} 1 & -1 & 1 \end{pmatrix}\) 
	\(B=\begin{pmatrix} 0 \\ 1 \\ -1 \end{pmatrix}\)
\end{enumerate}
\end{exercise}

\begin{exercise}
Для следующим матриц вычислите произведении \(A^\top B, AB^\top, B^\top A\) и \(BA^\top\), если операции определены
\begin{enumerate}
	\item \(A=\begin{pmatrix} 1 & 2 \\ 0 & -1 \end{pmatrix}\) 
	\(B=\begin{pmatrix} -1 & 1 \\ 2 & -2 \end{pmatrix}\)
	\item \(A=\begin{pmatrix} 1 & 1 & 0 \\ 0 & 1 & 1 \\ 1 & 0 & 1 \end{pmatrix}\) 
	\(B=\begin{pmatrix} -1 & 0 & 1 \\ 1 & -1 & 0 \\ 0 & 1 & -1 \end{pmatrix}\)
\end{enumerate}
\end{exercise}

\begin{exercise}
Рассмотрим матрицы
\begin{align*}
	A&=\begin{pmatrix}
		1 & -1 \\ -1 & 1 
	\end{pmatrix} &
	B&=\begin{pmatrix}
		-1 & 1 \\ 1 & -1 
	\end{pmatrix} &
		C&=\begin{pmatrix}
		1 & 1 \\ -1 & 1
	\end{pmatrix}
\end{align*}
Вычислите
\begin{align*}
	& AC-B & &BA+C & &(B+C)A & &C(A-B) & &AB-BC & &ABC
\end{align*}
\end{exercise}

\begin{exercise}
Рассмотрим матрицы
\begin{align*}
	A&=\begin{pmatrix}
		2 & 1 & -1 \\ -1 & 1 & 2 \\ 1 & 1 & -1 
	\end{pmatrix} &
	B&=\begin{pmatrix}
		-2 & 1 & 0 \\ 1 & -1 & 1 \\ 1 & -1 & 1 
	\end{pmatrix} &
		C&=\begin{pmatrix}
		1 & 2 & -1 \\ 2 & -1 & 1 \\ 1 & 1 & 0 
	\end{pmatrix}
\end{align*}
Вычислите
\begin{align*}
	& AB-C & &BC+A & &A(B+C) & &(2A-3B)C & &AB+BC & &ABC
\end{align*}
\end{exercise}

\begin{exercise}
Для матрицы \(A=\begin{pmatrix} 1 & 1 \\ 1 & 0 \end{pmatrix}\) вычислите
\(A^2, A^3, A^4\)
\end{exercise}
	
\begin{exercise}
Для матрицы \(A=\begin{pmatrix} 0 & 1 & -1 \\ -1 & 0 & 1 \\ 1 & -1 & 0 \end{pmatrix}\) вычислите
\(A^2, A^3, A^4\)
\end{exercise}

\subsubsection{Обратная матрица}

\begin{exercise}
Найдите обратную к следующим матрицам или покажите, что обратная не существует
\begin{align*}
	&\begin{pmatrix}
		1 & 1 \\ 0 & 1
	\end{pmatrix} &
	&\begin{pmatrix}
		1 & 1 \\ 1 & 0
	\end{pmatrix} &
	&\begin{pmatrix}
		0 & 1 \\ 1 & 0
	\end{pmatrix} &
	&\begin{pmatrix}
		2 & 1 \\ 3 & 0
	\end{pmatrix} &
	&\begin{pmatrix}
		1 & 1 \\ 2 & 2
	\end{pmatrix} \\
	&\begin{pmatrix}
		2 & 1 \\ 5 & 3
	\end{pmatrix} &
	&\begin{pmatrix}
		1 & 3 \\ 2 & 5
	\end{pmatrix} &
	&\begin{pmatrix}
		1 & 1 \\ 0 & 0
	\end{pmatrix} &
	&\begin{pmatrix}
		2 & 2 \\ 4 & 3
	\end{pmatrix} &
	&\begin{pmatrix}
		3 & 2 \\ 5 & 3
	\end{pmatrix} &
\end{align*}
\end{exercise}

\begin{exercise}
Найдите обратную к следующим матрицам или покажите, что обратная не существует
\begin{align*}
	&\begin{pmatrix}
		1 & 1 & 1 \\ 0 & 1 & 1 \\ 0 & 0 & 1
	\end{pmatrix} &
	&\begin{pmatrix}
		0 & 1 & -1 \\ -1 & 0 & 1 \\ 1 & -1 & 0
	\end{pmatrix} &
	&\begin{pmatrix}
		0 & 0 & 1 \\ 0 & 1 & 1 \\ 1 & 1 & 1
	\end{pmatrix} \\
	&\begin{pmatrix}
		1 & 0 & 1 \\ 1 & 1 & 1 \\ 1 & 0 & 1
	\end{pmatrix} &
	&\begin{pmatrix}
		3 & 0 & 2 \\ 0 & 1 & 1 \\ 5 & 0 & 3
	\end{pmatrix} &
	&\begin{pmatrix}
		1 & 1 & 0 \\ 0 & 1 & 1 \\ 0 & 1 & 1
	\end{pmatrix}
\end{align*}
\end{exercise}

\subsubsection{Матричные уравнения}

\begin{exercise}
Решите матричное уравнение \(AX=B\) для следующих матриц
% \begin{align*}
% 	1.\;& A=\begin{pmatrix} 1 & 1 \\ 0 & 1 \end{pmatrix}, B=\begin{pmatrix} 1 & 0 \\ 1 & 1 \end{pmatrix} &
% 	2.\;& A=\begin{pmatrix} 1 & 1 \\ 1 & 0 \end{pmatrix}, B=\begin{pmatrix} 1 & 2 & 1 \\ 1 & 1 & 0 \end{pmatrix}
% \end{align*}
\begin{enumerate}
	\item \(A=\begin{pmatrix}
		1 & 1 \\ 0 & 1
	\end{pmatrix}\), \(B=\begin{pmatrix}
		1 & 0 \\ 1 & 1
	\end{pmatrix}\)
	\item \(A=\begin{pmatrix}
		1 & 1 \\ 1 & 0
	\end{pmatrix}\), \(B=\begin{pmatrix}
		1 & 2 & 1 \\ 1 & 1 & 0
	\end{pmatrix}\)
	\item \(A=\begin{pmatrix}
		2 & 3 \\ 3 & 5
	\end{pmatrix}\), \(B=\begin{pmatrix}
		0 & -1 & 2 \\ -1 & 0 & 1 
	\end{pmatrix}\)
	\item \(A=\begin{pmatrix}
		3 & 2 \\ 5 & 3
	\end{pmatrix}\), \(B=\begin{pmatrix}
		0 & -1 & 2 & 1\\ 2 & 0 & 1 & -1
	\end{pmatrix}\)
\end{enumerate}
\end{exercise}

\begin{exercise}
Решите матричное уравнение \(AX=B\) для следующих матриц
\begin{enumerate}
	\item \(A=\begin{pmatrix}
		1 & 1 & 0 \\ 1 & 1 & 1 \\ 0 & 1 & 1
	\end{pmatrix}\), \(B=\begin{pmatrix}
		1 & 0 \\ 1 & 1 \\ 0 & 1
	\end{pmatrix}\)
	\item \(A=\begin{pmatrix}
		1 & 1 & -1 \\ 0 & 1 & 1 \\ 0 & 0 & 1
	\end{pmatrix}\), \(B=\begin{pmatrix}
		1 & 0 & -1 \\ 1 & 1 & 0\\ 0 & 1 & -1
	\end{pmatrix}\)
	\item \(A=\begin{pmatrix}
		0 & 1 & 0 \\ -1 & 0 & -1 \\ 0 & 1 & 0
	\end{pmatrix}\), \(B=\begin{pmatrix}
		1 & 2 & 0 & 1 \\ 1 & -1 & 1 & 0\\ 1 & 0 & 1 & 0
	\end{pmatrix}\)
\end{enumerate}
\end{exercise}

\begin{exercise}
Решите матричное уравнение \(XA=B\) для следующих матриц
\begin{enumerate}
	\item \(A=\begin{pmatrix}
		1 & 1 \\ 0 & 1
	\end{pmatrix}\), \(B=\begin{pmatrix}
		1 & 0 \\ 1 & 1
	\end{pmatrix}\)
	\item \(A=\begin{pmatrix}
		1 & 1 \\ 1 & 0
	\end{pmatrix}\), \(B=\begin{pmatrix}
		1 & 2 \\ 1 & 1 \\ 1 & 0
	\end{pmatrix}\)
	\item \(A=\begin{pmatrix}
		1 & 1 & 0 \\ 1 & 1 & 1 \\ 0 & 1 & 1
	\end{pmatrix}\), \(B=\begin{pmatrix}
		1 & 0 & -1 \\ 1 & 1 & 0\\ 0 & 1 & -1
	\end{pmatrix}\)
	\item \(A=\begin{pmatrix}
		0 & 1 & 0 \\ -1 & 0 & -1 \\ 0 & 1 & 0
	\end{pmatrix}\), \(B=\begin{pmatrix}
		1 & 0 & 1 \\ 1 & 1 & 0
	\end{pmatrix}\)
\end{enumerate}
\end{exercise}

\begin{exercise}
Решите матричное уравнение \(A_1XA_2=B\) для следующих матриц
\begin{enumerate}
	\item \(A_1=\begin{pmatrix}
		1 & 1 \\ 0 & 1
	\end{pmatrix}\), \(A_2=\begin{pmatrix}
		0 & 1 \\ 1 & 0
	\end{pmatrix}\), \(B=\begin{pmatrix}
		1 & 0 \\ 1 & 0
	\end{pmatrix}\)
	\item \(A_1=\begin{pmatrix}
		3 & 2 \\ 5 & 3
	\end{pmatrix}\), \(A_2=\begin{pmatrix}
		0 & 1 \\ 1 & 1
	\end{pmatrix}\), \(B=\begin{pmatrix}
		1 & 3 \\ 4 & 2
	\end{pmatrix}\)
	\item \(A_1=\begin{pmatrix}
		0 & 0 & 1 \\ 0 & 1 & -1 \\ 1 & 1 & 0
	\end{pmatrix}\), \(A_2=\begin{pmatrix}
		0 & 1 & 0 \\ 0 & 0 & 1 \\ 1 & 0 & 0
	\end{pmatrix}\), \(B=\begin{pmatrix}
		1 & 0 & -1 \\ 1 & 0 & 1 \\ 0 & 1 & 1
	\end{pmatrix}\)
	\item \(A_1=\begin{pmatrix}
		0 & -1 \\ 1 & 0
	\end{pmatrix}\), \(A_2=\begin{pmatrix}
		0 & 0 & 1 \\ 1 & 0 & 0 \\ 0 & 1 & 0
	\end{pmatrix}\), \(B=\begin{pmatrix}
		1 & 0 & 2 \\ 2 & 2 & 3
	\end{pmatrix}\)
	\item \(A_1=\begin{pmatrix}
		1 & 1 & 1 \\ 0 & 1 & 1 \\ 0 & 0 & 1
	\end{pmatrix}\), \(A_2=\begin{pmatrix}
		0 & 1 \\ 1 & 1
	\end{pmatrix}\), \(B=\begin{pmatrix}
		1 & 1 \\ 1 & 0 \\ 0 & 2
	\end{pmatrix}\)
\end{enumerate}
\end{exercise}

\subsubsection{Определитель}

\begin{exercise}
Вычислите определитель следующих матриц
\begin{align*}
	& \begin{pmatrix}
		2 & 5 \\ 1 & 3
	\end{pmatrix} &
	& \begin{pmatrix}
		3 & 2 \\ 5 & 3
	\end{pmatrix} &
	& \begin{pmatrix}
		2 & 5 \\ 1 & 3
	\end{pmatrix} &
	& \begin{pmatrix}
		2 & 3 \\ 6 & 9
	\end{pmatrix} \\
	& \begin{pmatrix}
		0 & 1 & 0 \\ 0 & 0 & 1 \\ 1 & 0 & 0
	\end{pmatrix} &
	& \begin{pmatrix}
		0 & 0 & 1 \\ 0 & 1 & 1 \\ 1 & 1 & 1
	\end{pmatrix} &
	& \begin{pmatrix}
		0 & 1 & 1 \\ 1 & 0 & 1 \\ 1 & 1 & 0
	\end{pmatrix} &
	& \begin{pmatrix}
		0 & 1 & 0 \\ -1 & 0 & -1 \\ 0 & 1 & 0
	\end{pmatrix}
\end{align*}
\end{exercise}

\begin{exercise}
Вычислите определитель следующих матриц
\begin{align*}
	& \begin{pmatrix}
		0 & 1 & 0 & 0 \\ 1 & 0 & 1 & 0 \\ 
		0 & 1 & 0 & 1 \\ 0 & 0 & 1 & 0 
	\end{pmatrix} &
	& \begin{pmatrix}
		0 & 0 & 0 & 1 \\ 0 & 0 & 1 & 1 \\ 
		0 & 1 & 1 & 1 \\ 1 & 1 & 1 & 1
	\end{pmatrix} &
	& \begin{pmatrix}
		0 & 1 & 1 & 1\\ 1 & 0 & 1 & 1\\ 
		1 & 1 & 0 & 1 \\ 1 & 1 & 1 & 0
	\end{pmatrix} &
	& \begin{pmatrix}
		0 & 1 & 0 & 0 \\ -1 & 0 & -1 & 0 \\ 
		0 & 1 & 0 & 1 \\ 0 & 0 & -1 & 0
	\end{pmatrix}
\end{align*}
\end{exercise}

\begin{exercise}
Вычислите определитель следующих матриц
\begin{align*}
	& \begin{pmatrix}
		0 & 1 & 0 & 0 & 0 \\ 1 & 0 & 1 & 0 & 0\\ 
		0 & 1 & 0 & 1 & 0 \\ 0 & 0 & 1 & 0 & 1 \\
		0 & 0 & 0 & 1 & 0
	\end{pmatrix} &
	& \begin{pmatrix}
		0 & 1 & 0 & 0 & 0 \\ -1 & 0 & 1 & 0 & 0\\ 
		0 & -1 & 0 & 1 & 0 \\ 0 & 0 & -1 & 0 & 1 \\
		0 & 0 & 0 & -1 & 0
	\end{pmatrix} &
	& \begin{pmatrix}
		0 & 1 & 1 & 1 & 1\\ 1 & 0 & 1 & 1 & 1\\ 
		1 & 1 & 0 & 1 & 1\\ 1 & 1 & 1 & 0 & 1\\
		1 & 1 & 1 & 1 & 0 
	\end{pmatrix} 
\end{align*}
\end{exercise}

\begin{exercise}
Какие из следующи матриц обратимы?
\begin{align*}
	& \begin{pmatrix}
		2 & 5 \\ 1 & 3
	\end{pmatrix} &
	& \begin{pmatrix}
		3 & 2 \\ 5 & 3
	\end{pmatrix} &
	& \begin{pmatrix}
		2 & 5 \\ 4 & 10
	\end{pmatrix} &
	& \begin{pmatrix}
		6 & 9 \\ 4 & 6
	\end{pmatrix} &
	& \begin{pmatrix}
		3 & 5 \\ 4 & 7
	\end{pmatrix}
\end{align*}
\end{exercise}

\begin{exercise}
Какие из следующи матриц обратимы?
\begin{align*}
		& \begin{pmatrix}
			1 & 0 & 1 \\ 0 & 1 & 0 \\ 1 & 0 & 1
		\end{pmatrix} &
		& \begin{pmatrix}
			0 & 1 & 1 \\ 0 & 0 & 1 \\ 0 & 0 & 0
		\end{pmatrix} &
		& \begin{pmatrix}
			0 & 1 & -1 \\ -1 & 0 & 1 \\ 1 & -1 & 0
		\end{pmatrix} &
		& \begin{pmatrix}
			6 & 9 & 3 \\ 4 & 6 & 2 \\ 1 & 0 & 1
		\end{pmatrix} 
\end{align*}
\end{exercise}

\subsection{Системы линейных уравнений}

\begin{exercise}
Рассмотрим систему линейных уравнений в матричном виде \(Ax=b\).
Для следующих матриц запишите систему линейных уравнений и решите её использую обратную матрицу
\begin{align*}
	A,b&=\begin{pmatrix}
		2 & 5 \\ 1 & 3
	\end{pmatrix}, \begin{pmatrix}
		2 \\ 4
	\end{pmatrix} & 
	A,b&=\begin{pmatrix}
		3 & 4 \\ 2 & 3
	\end{pmatrix}, \begin{pmatrix}
		3 \\ 5
	\end{pmatrix} & 
	A,b&=\begin{pmatrix}
		5 & 3 \\ 7 & 4
	\end{pmatrix}, \begin{pmatrix}
		-1 \\ 0
	\end{pmatrix}
\end{align*}
\end{exercise}

\begin{exercise}
Рассмотрим систему линейных уравнений в матричном виде \(Ax=b\).
Для следующих матриц запишите систему линейных уравнений и решите её использую обратную матрицу
\begin{align*}
	A,b&=\begin{pmatrix}
		0 & 1 & 1 \\ 1 & 0 & 1 \\ 1 & 1 & 0
	\end{pmatrix}, \begin{pmatrix}
		1 \\ 0 \\ 1
	\end{pmatrix} & 
	A,b&=\begin{pmatrix}
		1 & 2 & 3 \\ 0 & -1 & 2 \\ 0 & 0 & 1
	\end{pmatrix}, \begin{pmatrix}
		2 \\ 1 \\ 1
	\end{pmatrix} \\
	A,b&=\begin{pmatrix}
		5 & 0 & 2 \\ 0 & 3 & 0 \\ 3 & 0 & 1
	\end{pmatrix}, \begin{pmatrix}
		1 \\ 6 \\ 1
	\end{pmatrix} & 
	A,b&=\begin{pmatrix}
		3 & 4 & 0 \\ 5 & 7 & 1 \\ 0 & 0 & 2
	\end{pmatrix}, \begin{pmatrix}
		1 \\ 2 \\ 2
	\end{pmatrix}
\end{align*}
\end{exercise}

\begin{exercise}
Решите систему линейных уравнений использую формулы Крамера
\begin{align*}
	&\left\{\begin{aligned}
		2x_1-3x_2&=0 \\ 3x_1+5x_2 &= 2
	\end{aligned}\right. & 
	&\left\{\begin{aligned}
		x_1-x_2+x_3&=1 \\ 2x_1+2x_2-3x_3 &= 0 \\
		3x_1+x_2-4x_3 &= 2
	\end{aligned}\right. \\
	&\left\{\begin{aligned}
		2x_1-3x_2&=0 \\ 3x_1+5x_2 &= 2
	\end{aligned}\right. &
	&\left\{\begin{aligned}
		x_1+x_2-x_3&=1 \\ x_1-2x_2+3x_3 &= 2 \\
		2x_1-x_2-3x_3 &= 5
	\end{aligned}\right.
\end{align*}
\end{exercise}

\begin{exercise}
Какие из следующих систем \(Ax=b\) имеют единственной решений?
\begin{align*}
	1)\;A,b&=\begin{pmatrix}
		4 & 6  \\ 6 & 9 
	\end{pmatrix}, \begin{pmatrix}
		0 \\ 1 
	\end{pmatrix} &
	2)\;A,b&=\begin{pmatrix}
		0 & 1 & 0 \\ -1 & 0 & 1 \\ 0 & -1 & 0
	\end{pmatrix}, \begin{pmatrix}
		2 \\ 1 \\ 4
	\end{pmatrix} \\
	3)\;A,b&=\begin{pmatrix}
		4 & 7  \\ 3 & 5
	\end{pmatrix}, \begin{pmatrix}
		1 \\ 2 
	\end{pmatrix} &
	4)\;A,b&=\begin{pmatrix}
		2 & 3 & 1 \\ 5 & 7 & -2 \\ 0 & 0 & -1
	\end{pmatrix}, \begin{pmatrix}
		2 \\ 1 \\ 4
	\end{pmatrix}
\end{align*}
\end{exercise}