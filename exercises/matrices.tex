% !TEX root = exercises-math-for-ds.tex

\subsection{Операции с матрицами}

\subsubsection{Скалярное умножение и сложение}

\begin{exercise}
Рассмотрим матрицы
\begin{align*}
	A&=\begin{pmatrix}
		-1 & 2 & 0 \\ 0 & 2 & 3 \\ 1 & -1 & 0 \\ 2 & -2 & 0
	\end{pmatrix} &
	B&=\begin{pmatrix}
		0 & -1 & 1 \\ 0 & 1 & 1 \\ 2 & 0 & -1 \\ 1 & 1 & 2
	\end{pmatrix} &
	C&=\begin{pmatrix}
		2 & 0 & -1 \\ -1 & -2 & 2 \\ 1 & -1& 2 \\ 0 & 3 & -1
	\end{pmatrix}
\end{align*}
Вычислите
\begin{align*}
	& 2A+B & &A-2C & &4B-A-C & &C-2A+4B
\end{align*}
\end{exercise}

\begin{exercise}
Рассмотрим матрицы
\begin{align*}
	A&=\begin{pmatrix}
		2 & 1 & 5 \\ 3 & 4 & 3 \\ 1 & 2 & 0 \\ 2 & 3 & 1 \\ 1 & 1 & 0
	\end{pmatrix} &
	B&=\begin{pmatrix}
		2 & 1 & 0 \\ 2 & 5 & 2 \\ 4 & 3 & 2 \\ 3 & 4 & 1 \\ 1 & 3 & 2
	\end{pmatrix} &
	C&=\begin{pmatrix}
		5 & 2 & 3 \\ 2 & 3 & 0 \\ 2 & 1 & 0 \\ 1 & 0 & 1 \\ 2 & 2 & 3
	\end{pmatrix}
\end{align*}
Вычислите
\begin{align*}
	& A+3B & &3B-2C & &2B-C+3A & &2C+3A-5B
\end{align*}
\end{exercise}

\begin{exercise}
Рассмотрим матрицы
\begin{align*}
	A&=\begin{pmatrix}
		-1 & 2 & 2 & 1 & 0 \\ 1 & 0 & -2 & 1 & 0
	\end{pmatrix} \\
	B&=\begin{pmatrix}
		0 & 0 & 1 & 3 & 2 \\ -1 & 0 & 2 & 1 & -3
	\end{pmatrix} \\
	C&=\begin{pmatrix}
		1 & 2 & 0 & 1 & 0 \\ 0 & 1 & 1 & -1 & -2
	\end{pmatrix}
\end{align*}
Вычислите
\begin{align*}
	& 3A-B & &2A-C & &2B-C+3A & &B-2A+C
\end{align*}
\end{exercise}

\subsubsection{Умножение метриц}

\textbf{Замечание}: через \(\odot\) будем обозначать \textit{произведение Адамара} для матриц

\begin{exercise}
Для следующим матриц вычислите \(A\odot B\) если операция определение
\begin{align*}
	&(1) & A&=\begin{pmatrix}
		1 & 2 \\ 0 & -1
	\end{pmatrix} & B&=\begin{pmatrix}
		-1 & 1 \\ 2 & -2
	\end{pmatrix} \\
	&(2) & A&=\begin{pmatrix}
		1 & 1 & 0 \\ 0 & 1 & 1 \\ 1 & 0 & 1 
	\end{pmatrix} & B&=\begin{pmatrix}
		-1 & 1 \\ 2 & -2
	\end{pmatrix}
\end{align*}
\end{exercise}