% !TEX root = exercises-math-for-ds.tex

\subsection{Функции одной переменной}

\begin{exercise}
Вычислите первую производную функций
\begin{align*}
	f(x)&=x\cos(x) & f(x)&=x\sin(x) & f(x)&=x^2\sin(x) & f(x)&=x^2\cos(x) \\
	f(x)&=\cos^2(x) & f(x)&=\sin^2(x) & f(x)&=x\cos^2(x) & f(x)&=x\sin^2(x) \\
	f(x)&=\frac{\sin(x)}{x} & f(x)&=\frac{\cos(x)}{x} & f(x)&=\frac{\cos^2{x}}{x} & f(x)&=\frac{\sin{x}}{x^2} \\
	f(x)&=x\ln x & f(x)&=x^2\ln x & f(x)&=x\ln^2(x) & f(x)&=\frac{\ln x}{x} \\
	f(x)&=x\exp(x) & f(x)&=\exp(x^2) & f(x)&=x\exp(-x) & f(x)&=x\exp(-x^2) \\
\end{align*}
\end{exercise}

\begin{exercise}
Вычислите значение первой производной функции
\begin{enumerate}
	\item \(f(x)=x\cos(x)\) в точках \(x=0, \pi/2, \pi\)
	\item \(f(x)=x^2\sin(x)\) в точках \(x=1, \pi/2, \pi\)
	\item \(f(x)=x^3\ln x\) в точках \(x=1, 2, 3\)
	\item \(f(x)=x\exp(x^2)\) в точках \(x=1, 2, 3\)
\end{enumerate}
\end{exercise}

\begin{exercise}
Вычислите вторую производную функций
\begin{align*}
	f(x)&=x\cos(x) & f(x)&=x\sin(x) & f(x)&=\cos^2(x) & f(x)&=\sin^2(x) \\
	f(x)&=x\ln x & f(x)&=x\exp(x) & f(x)&=x^2\exp(-x) & f(x)&=\exp(-x^2)
\end{align*}
\end{exercise}

\begin{exercise}
Найдите (численно) локальные экстремумы функции 
\begin{align*}
	f(x)&=10+3x-x^2 & f(x)&=2x^2+4x-5 \\
	f(x)&=x^3-4x^2+3x-10 & f(x)&=6+3x-5x^2-x^3 \\
	f(x)&=x\exp(x) & f(x)&=x^2\exp(x) \\
	f(x)&=x^3\exp(-x) & f(x)&=x\exp(-x^2) 
\end{align*}
\end{exercise}

\subsection{Функции многих переменных}

\begin{exercise}
Вычислите градиент следующих функции
\begin{align*}
	f&=xy & f&=x^2y^2 &  f&=x^2y-xy^2 & f&=x^2-xy+y^2 \\
	f&=\exp(xy) & f&=\exp(x+y) & f&=\ln(x+y) & f&=\exp(x^2y)
\end{align*}
\end{exercise}

\begin{exercise}
Найдите значение градиента функции
\begin{enumerate}
	\item \(f=xy^2\) в точке \((1,2)\)
	\item \(f=x^2y+xy^2\) в точке \((2,-1)\)
	\item \(f=x^2+xy+y^2\) в точке \((-1,2)\)
	\item \(f=\ln(x^2+y^2)\) в точке \((2,3)\)
	\item \(f=\exp(x^2+y^2)\) в точке \((-2,1)\)
\end{enumerate}
\end{exercise}

\begin{exercise}
Найдите локальные экстремумы функций
\begin{align*}
	f(x,y) &= 10-6x-4y+2x^2+y^2-2xy \\
	f(x,y) &= 8+8x+4y-5x^2-2y^2+6xy \\
	f(x,y) &= 5+2x+6y+5x^2+3y^2+8xy
\end{align*}
% Нарисуйте графики функций (MS Excel, Python etc)
\end{exercise}

\begin{exercise}
Найдите локальные экстремумы функций
\begin{align*}
	f(x,y,z) &= 6+4x+2y+6z+2x^2+2y^2+z^2+2xy+2yz \\
	f(x,y,z) &= 3+4x+8y+4z-3x^2-2y^2-4z^2+2xy+2xz+4yz\\
	f(x,y,z) &= 8+2x+4y+2z+2x^2+y^2+3z^2+2xy+4xz+4yz
\end{align*}
\end{exercise}
	
\begin{exercise}
Найдите локальные экстремумы функций
\begin{align*}
	f(x,y) &= 5+x^3-y^3+3xy \\
	f(x,y) &= 3x^2y+y^3-3x^2-3y^2+2 \\
	f(x,y) &= x^3+x^2y-2y^3+6y
\end{align*}
\end{exercise}

\begin{exercise}
Найдите локальные экстремумы функций
\begin{align*}
	f(x,y) &= 6\ln x+8\ln y-3x-2y \\
	f(x,y) &= 4\ln x+6\ln y+2x-3xy \\
	f(x,y) &= 5\ln x+4\ln y-x-4xy
\end{align*}
\end{exercise}