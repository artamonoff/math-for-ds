\documentclass[12pt]{article}

\usepackage[utf8]{inputenc}
\usepackage[T2A]{fontenc}
\usepackage[english, russian]{babel}
\usepackage{amsmath, amsthm, amssymb}
\usepackage{enumerate,hhline, bm}
\usepackage[mathscr]{eucal}
%\usepackage{mathtext}

\usepackage{hyperref}
\hypersetup{unicode=true, final=true, colorlinks=true}


% 
%   Вероятностные определения
%
\DeclareMathOperator{\cov}{cov}
\DeclareMathOperator{\corr}{corr}
\DeclareMathOperator*{\plim}{plim}
\DeclareMathOperator{\Var}{Var}
\DeclareMathOperator{\VVar}{V}
\newcommand{\StdDev}{s.d.}

%
%  Эконометрические
%
\DeclareMathOperator{\const}{const}
\DeclareMathOperator{\error}{error}

%
%  Линейная алгебра
%
\DeclareMathOperator{\rank}{rank}
\DeclareMathOperator{\dimension}{dim}
\DeclareMathOperator{\tr}{tr}
\newcommand{\LinearSpace}{{\mathfrak L}}
\newcommand{\spaceX}{{\mathbb X}}
\newcommand{\spaceY}{{\mathbb Y}}


%
%   Числовые
%
\newcommand{\Complex}{{\mathbb C}}
\newcommand{\N}{\mathbb N}
\newcommand{\Z}{{\mathbb Z}}
\newcommand{\Q}{{\mathbb Q}}
\newcommand{\R}{{\mathbb R}}
\newcommand{\semiaxes}{{\mathbb R_+}}

%
%  Вероятностные
%
\newcommand{\iid}{{i.i.d.}}
\newcommand{\Exp}{{\mathsf E}}
\newcommand{\Gauss}{{\mathscr N}}
\newcommand{\Likelihood}{{\mathcal L}}
\newcommand{\StError}{{s.e.}}
\newcommand{\ConfInterval}{{\mathcal I}}

%
%   Вектора
%
\newcommand{\vconst}{{\mathbf const}}
\newcommand{\vectx}{{\bm x}}
\newcommand{\vecty}{{\bm y}}
\newcommand{\vectz}{{\bm z}}
\newcommand{\vecte}{{\bm e}}
\newcommand{\vectw}{{\bm w}}
\newcommand{\vecth}{{\bm h}}
\newcommand{\vectr}{{\bm r}}
\newcommand{\vectq}{{\bm q}}
\newcommand{\vectf}{{\bm f}}%{\boldsymbol{f}}
\newcommand{\vectu}{{\bm u}}
\newcommand{\vectv}{{\bm v}}
\newcommand{\vectalpha}{{\bm{\alpha}}}
\newcommand{\vectbeta}{{\bm{\beta}}}
\newcommand{\vectgamma}{{\bm{\gamma}}}
\newcommand{\vectdelta}{{\bm{\delta}}}
\newcommand{\vectomega}{{\bm{\omega}}}
\newcommand{\vecttheta}{{\bm{\theta}}}
\newcommand{\vecteta}{{\bm{\eta}}}
\newcommand{\vectpi}{{\bm{\pi}}}
\newcommand{\vectmu}{{\bm{\mu}}}
\newcommand{\vectxi}{{\bm{\xi}}}
\newcommand{\vectX}{{\bm X}}
\newcommand{\vectY}{{\bm Y}}
\newcommand{\vectZ}{{\bm Z}}
\newcommand{\vectones}{{ 1}}

% 
%  Матрицы
%
\newcommand{\Id}{I}
\newcommand{\matrixX}{{\bm X}}
\newcommand{\matrixY}{{\bm Y}}
\newcommand{\matrixU}{{\bm U}}
\newcommand{\matrixV}{{\bm V}}
\newcommand{\matrixR}{{\bm R}}
\newcommand{\matrixZ}{{\bm Z}}
\newcommand{\matrixA}{{\bm A}}
\newcommand{\matrixB}{{\bm B}}
\newcommand{\matrixQ}{{\bm Q}}
\newcommand{\matrixH}{{\bm H}}
\newcommand{\matrixXX}{X}
\newcommand{\matrixGamma}{{\bm{\Gamma}}}
\newcommand{\matrixPi}{{\bm{\Pi}}}

%
% Теоремы, Примеры etc
%
\newtheorem*{teorema}{Теорема}
\newtheorem*{importante}{Важно!}
\newtheorem*{ejemplo}{Пример}
\newtheorem*{definicion}{Определение}
\newtheorem*{proposition}{Предложение}

\theoremstyle{remark}
\newtheorem*{remark}{Замечание}


% \theoremstyle{plain}
% \newtheorem*{trm}{Теорема}
% \newtheorem*{mprtnt}{Важно!}
% \newtheorem*{xmpl}{Пример}
% \newtheorem*{dfntn}{Определение}

% \theoremstyle{remark}
% \newtheorem*{rmrk}{Замечание}

\theoremstyle{remark}
\newtheorem{exercise}{}[subsection]
\renewcommand{\theexercise}{\textbf{\textnumero \arabic{exercise}}}

%\DeclareMathOperator{\cov}{cov}
%\DeclareMathOperator{\corr}{corr}
%\DeclareMathOperator{\Var}{Var}

%\topmargin=-2cm%-1.5cm

%\addtolength{\textheight}{3cm}

%\oddsidemargin=-0.1cm

%\addtolength{\textwidth}{1.8cm}


\title{Задачи по Математическим основам анализа данных}
\author{Артамонов Н.В.}
%\date{весна 2014}

%\title{Задачи для подготовки к экзамену по курсу 
%<<Методы оптимальных решений>>}\author{\copyright Артамонов Н.В., кафедра ЭММАЭ}

\begin{document}

\maketitle

%\markright{}
\tableofcontents

\section{Работа с массивами (матричный анализ)}

% !TEX root = exercises-math-for-ds.tex

\subsection{Операции с матрицами}

\subsubsection{Скалярное умножение и сложение}

\begin{exercise}
Рассмотрим матрицы
\begin{align*}
	A&=\begin{pmatrix}
		-1 & 2 & 0 \\ 0 & 2 & 3 \\ 1 & -1 & 0 \\ 2 & -2 & 0
	\end{pmatrix} &
	B&=\begin{pmatrix}
		0 & -1 & 1 \\ 0 & 1 & 1 \\ 2 & 0 & -1 \\ 1 & 1 & 2
	\end{pmatrix} &
	C&=\begin{pmatrix}
		2 & 0 & -1 \\ -1 & -2 & 2 \\ 1 & -1& 2 \\ 0 & 3 & -1
	\end{pmatrix}
\end{align*}
Вычислите
\begin{align*}
	& 2A+B & &A-2C & &4B-A-C & &C-2A+4B
\end{align*}
\end{exercise}

\begin{exercise}
Рассмотрим матрицы
\begin{align*}
	A&=\begin{pmatrix}
		2 & 1 & 5 \\ 3 & 4 & 3 \\ 1 & 2 & 0 \\ 2 & 3 & 1 \\ 1 & 1 & 0
	\end{pmatrix} &
	B&=\begin{pmatrix}
		2 & 1 & 0 \\ 2 & 5 & 2 \\ 4 & 3 & 2 \\ 3 & 4 & 1 \\ 1 & 3 & 2
	\end{pmatrix} &
	C&=\begin{pmatrix}
		5 & 2 & 3 \\ 2 & 3 & 0 \\ 2 & 1 & 0 \\ 1 & 0 & 1 \\ 2 & 2 & 3
	\end{pmatrix}
\end{align*}
Вычислите
\begin{align*}
	& A+3B & &3B-2C & &2B-C+3A & &2C+3A-5B
\end{align*}
\end{exercise}

\begin{exercise}
Рассмотрим матрицы
\begin{align*}
	A&=\begin{pmatrix}
		-1 & 2 & 2 & 1 & 0 \\ 1 & 0 & -2 & 1 & 0
	\end{pmatrix} \\
	B&=\begin{pmatrix}
		0 & 0 & 1 & 3 & 2 \\ -1 & 0 & 2 & 1 & -3
	\end{pmatrix} \\
	C&=\begin{pmatrix}
		1 & 2 & 0 & 1 & 0 \\ 0 & 1 & 1 & -1 & -2
	\end{pmatrix}
\end{align*}
Вычислите
\begin{align*}
	& 3A-B & &2A-C & &2B-C+3A & &B-2A+C
\end{align*}
\end{exercise}

\subsubsection{Умножение метриц}

\textbf{Замечание}: через \(\odot\) будем обозначать \textit{произведение Адамара} для матриц

\begin{exercise}
Для следующим матриц вычислите \(A\odot B\), если операция определена
\begin{enumerate}
	\item \(A=\begin{pmatrix} 1 & 2 \\ 0 & -1 \end{pmatrix}\) 
	\(B=\begin{pmatrix} -1 & 1 \\ 2 & -2 \end{pmatrix}\)
	\item \(A=\begin{pmatrix} 1 & 1 & 0 \\ 0 & 1 & 1 \\ 1 & 0 & 1 \end{pmatrix}\) 
	\(B=\begin{pmatrix} -1 & 0 & 1 \\ 1 & -1 & 0 \\ 0 & 1 & -1 \end{pmatrix}\)
	\item \(A=\begin{pmatrix} 1 & 0 & -1 & 1\\ 1 & 2 & -1 & 0  \end{pmatrix}\) 
	\(B=\begin{pmatrix} -1 & 1 & 1 & 2 \\ 0 & 1 & 2 & -2 \end{pmatrix}\)
	\item \(A=\begin{pmatrix} 1 \\ -1 \\ 2 \\ 0 \\ -2  \end{pmatrix}\) 
	\(B=\begin{pmatrix} 0 \\ 2 \\ -1 \\ 1 \\ 0  \end{pmatrix}\)
\end{enumerate}
\end{exercise}

\begin{exercise}
Для матрицы \(A=\begin{pmatrix} 1 & 1 \\ 0 & 1 \end{pmatrix}\) вычислите
\begin{align*}
	&A\odot A & &A^\top\odot A & &A\odot A\odot A & &A\odot A^\top\odot A & &A\odot A^\top\odot A^\top
\end{align*}
\end{exercise}

\begin{exercise}
Для матрицы \(A=\begin{pmatrix} 1 & 1 & 0 \\ 0 & 1 & 1 \\ 1 & 1 & 0 \end{pmatrix}\) вычислите
\begin{align*}
	&A\odot A & &A^\top\odot A & &A\odot A\odot A & &A\odot A^\top\odot A & &A\odot A^\top\odot A^\top
\end{align*}
\end{exercise}

\begin{exercise}
Рассмотрим матрицы
\begin{align*}
	A&=\begin{pmatrix}
		-1 & 2 & 0 \\ 0 & 2 & 3 \\ 1 & -1 & 0 \\ 2 & -2 & 0
	\end{pmatrix} &
	B&=\begin{pmatrix}
		0 & -1 & 1 \\ 0 & 1 & 1 \\ 2 & 0 & -1 \\ 1 & 1 & 2
	\end{pmatrix} &
	C&=\begin{pmatrix}
		2 & 0 & -1 \\ -1 & -2 & 2 \\ 1 & -1& 2 \\ 0 & 3 & -1
	\end{pmatrix}
\end{align*}
Вычислите
\begin{align*}
	& A\odot B\odot C & &A\odot B-C & &2B\odot C-A& &2A\odot B-3B\odot C
\end{align*}
\end{exercise}

\begin{exercise}
Рассмотрим матрицы
\begin{align*}
	A&=\begin{pmatrix}
		2 & 1 & 5 \\ 3 & 4 & 3 \\ 1 & 2 & 0 \\ 2 & 3 & 1 \\ 1 & 1 & 0
	\end{pmatrix} &
	B&=\begin{pmatrix}
		2 & 1 & 0 \\ 2 & 5 & 2 \\ 4 & 3 & 2 \\ 3 & 4 & 1 \\ 1 & 3 & 2
	\end{pmatrix} &
		C&=\begin{pmatrix}
		5 & 2 & 3 \\ 2 & 3 & 0 \\ 2 & 1 & 0 \\ 1 & 0 & 1 \\ 2 & 2 & 3
	\end{pmatrix}
\end{align*}
Вычислите
\begin{align*}
	& A\odot B\odot C & &2A\odot B-C & &B\odot C+2A& &3A\odot B-2B\odot C
\end{align*}
\end{exercise}
	
\begin{exercise}
Рассмотрим матрицы
\begin{align*}
	A&=\begin{pmatrix}
		-1 & 2 & 2 & 1 & 0 \\ 1 & 0 & -2 & 1 & 0
	\end{pmatrix} \\
	B&=\begin{pmatrix}
		0 & 0 & 1 & 3 & 2 \\ -1 & 0 & 2 & 1 & -3
	\end{pmatrix} \\
	C&=\begin{pmatrix}
		1 & 2 & 0 & 1 & 0 \\ 0 & 1 & 1 & -1 & -2
	\end{pmatrix}
\end{align*}
Вычислите
\begin{align*}
	& & A\odot B\odot C & &2A\odot C-B & &B\odot C-2B& &3A\odot C-2A\odot C
\end{align*}
\end{exercise}

\begin{exercise}
Для следующим матриц вычислите произведении \(AB\) и \(BA\), если операции определены
\begin{enumerate}
	\item \(A=\begin{pmatrix} 1 & 2 \\ 0 & -1 \end{pmatrix}\) 
	\(B=\begin{pmatrix} -1 & 1 \\ 2 & -2 \end{pmatrix}\)
	\item \(A=\begin{pmatrix} 1 & 2 \\ 0 & -1 \end{pmatrix}\) 
	\(B=\begin{pmatrix} 1 & 1 & -1 \\ 0 & 2 & 1 \end{pmatrix}\)
	\item \(A=\begin{pmatrix} 1 & 1 & 0 \\ 0 & 1 & 1 \\ 1 & 0 & 1 \end{pmatrix}\) 
	\(B=\begin{pmatrix} -1 & 0 & 1 \\ 1 & -1 & 0 \\ 0 & 1 & -1 \end{pmatrix}\)
	\item \(A=\begin{pmatrix} 1 & 0 & -1 & 1\\ 1 & 1 & -1 & 0  \end{pmatrix}\) 
	\(B=\begin{pmatrix} -1 & 1 \\ 1 & -1 \\ 0 & 1 \\ 1 & -1 \end{pmatrix}\)
	\item \(A=\begin{pmatrix} 1 & -1 & 1 \end{pmatrix}\) 
	\(B=\begin{pmatrix} 0 \\ 1 \\ -1 \end{pmatrix}\)
\end{enumerate}
\end{exercise}

\begin{exercise}
Для следующим матриц вычислите произведении \(A^\top B, AB^\top, B^\top A\) и \(BA^\top\), если операции определены
\begin{enumerate}
	\item \(A=\begin{pmatrix} 1 & 2 \\ 0 & -1 \end{pmatrix}\) 
	\(B=\begin{pmatrix} -1 & 1 \\ 2 & -2 \end{pmatrix}\)
	\item \(A=\begin{pmatrix} 1 & 1 & 0 \\ 0 & 1 & 1 \\ 1 & 0 & 1 \end{pmatrix}\) 
	\(B=\begin{pmatrix} -1 & 0 & 1 \\ 1 & -1 & 0 \\ 0 & 1 & -1 \end{pmatrix}\)
\end{enumerate}
\end{exercise}

\begin{exercise}
Рассмотрим матрицы
\begin{align*}
	A&=\begin{pmatrix}
		1 & -1 \\ -1 & 1 
	\end{pmatrix} &
	B&=\begin{pmatrix}
		-1 & 1 \\ 1 & -1 
	\end{pmatrix} &
		C&=\begin{pmatrix}
		1 & 1 \\ -1 & 1
	\end{pmatrix}
\end{align*}
Вычислите
\begin{align*}
	& AC-B & &BA+C & &(B+C)A & &C(A-B) & &AB-BC & &ABC
\end{align*}
\end{exercise}

\begin{exercise}
Рассмотрим матрицы
\begin{align*}
	A&=\begin{pmatrix}
		2 & 1 & -1 \\ -1 & 1 & 2 \\ 1 & 1 & -1 
	\end{pmatrix} &
	B&=\begin{pmatrix}
		-2 & 1 & 0 \\ 1 & -1 & 1 \\ 1 & -1 & 1 
	\end{pmatrix} &
		C&=\begin{pmatrix}
		1 & 2 & -1 \\ 2 & -1 & 1 \\ 1 & 1 & 0 
	\end{pmatrix}
\end{align*}
Вычислите
\begin{align*}
	& AB-C & &BC+A & &A(B+C) & &(2A-3B)C & &AB+BC & &ABC
\end{align*}
\end{exercise}

\begin{exercise}
Для матрицы \(A=\begin{pmatrix} 1 & 1 \\ 1 & 0 \end{pmatrix}\) вычислите
\(A^2, A^3, A^4\)
\end{exercise}
	
\begin{exercise}
Для матрицы \(A=\begin{pmatrix} 0 & 1 & -1 \\ -1 & 0 & 1 \\ 1 & -1 & 0 \end{pmatrix}\) вычислите
\(A^2, A^3, A^4\)
\end{exercise}

\subsubsection{Обратная матрица}

\begin{exercise}
Найдите обратную к следующим матрицам или покажите, что обратная не существует
\begin{align*}
	&\begin{pmatrix}
		1 & 1 \\ 0 & 1
	\end{pmatrix} &
	&\begin{pmatrix}
		1 & 1 \\ 1 & 0
	\end{pmatrix} &
	&\begin{pmatrix}
		0 & 1 \\ 1 & 0
	\end{pmatrix} &
	&\begin{pmatrix}
		2 & 1 \\ 3 & 0
	\end{pmatrix} &
	&\begin{pmatrix}
		1 & 1 \\ 2 & 2
	\end{pmatrix} \\
	&\begin{pmatrix}
		2 & 1 \\ 5 & 3
	\end{pmatrix} &
	&\begin{pmatrix}
		1 & 3 \\ 2 & 5
	\end{pmatrix} &
	&\begin{pmatrix}
		1 & 1 \\ 0 & 0
	\end{pmatrix} &
	&\begin{pmatrix}
		2 & 2 \\ 4 & 3
	\end{pmatrix} &
	&\begin{pmatrix}
		3 & 2 \\ 5 & 3
	\end{pmatrix} &
\end{align*}
\end{exercise}

\begin{exercise}
Найдите обратную к следующим матрицам или покажите, что обратная не существует
\begin{align*}
	&\begin{pmatrix}
		1 & 1 & 1 \\ 0 & 1 & 1 \\ 0 & 0 & 1
	\end{pmatrix} &
	&\begin{pmatrix}
		0 & 1 & -1 \\ -1 & 0 & 1 \\ 1 & -1 & 0
	\end{pmatrix} &
	&\begin{pmatrix}
		0 & 0 & 1 \\ 0 & 1 & 1 \\ 1 & 1 & 1
	\end{pmatrix} \\
	&\begin{pmatrix}
		1 & 0 & 1 \\ 1 & 1 & 1 \\ 1 & 0 & 1
	\end{pmatrix} &
	&\begin{pmatrix}
		3 & 0 & 2 \\ 0 & 1 & 1 \\ 5 & 0 & 3
	\end{pmatrix} &
	&\begin{pmatrix}
		1 & 1 & 0 \\ 0 & 1 & 1 \\ 0 & 1 & 1
	\end{pmatrix}
\end{align*}
\end{exercise}

\subsubsection{Матричные уравнения}

\begin{exercise}
Решите матричное уравнение \(AX=B\) для следующих матриц
% \begin{align*}
% 	1.\;& A=\begin{pmatrix} 1 & 1 \\ 0 & 1 \end{pmatrix}, B=\begin{pmatrix} 1 & 0 \\ 1 & 1 \end{pmatrix} &
% 	2.\;& A=\begin{pmatrix} 1 & 1 \\ 1 & 0 \end{pmatrix}, B=\begin{pmatrix} 1 & 2 & 1 \\ 1 & 1 & 0 \end{pmatrix}
% \end{align*}
\begin{enumerate}
	\item \(A=\begin{pmatrix}
		1 & 1 \\ 0 & 1
	\end{pmatrix}\), \(B=\begin{pmatrix}
		1 & 0 \\ 1 & 1
	\end{pmatrix}\)
	\item \(A=\begin{pmatrix}
		1 & 1 \\ 1 & 0
	\end{pmatrix}\), \(B=\begin{pmatrix}
		1 & 2 & 1 \\ 1 & 1 & 0
	\end{pmatrix}\)
	\item \(A=\begin{pmatrix}
		2 & 3 \\ 3 & 5
	\end{pmatrix}\), \(B=\begin{pmatrix}
		0 & -1 & 2 \\ -1 & 0 & 1 
	\end{pmatrix}\)
	\item \(A=\begin{pmatrix}
		3 & 2 \\ 5 & 3
	\end{pmatrix}\), \(B=\begin{pmatrix}
		0 & -1 & 2 & 1\\ 2 & 0 & 1 & -1
	\end{pmatrix}\)
\end{enumerate}
\end{exercise}

\begin{exercise}
Решите матричное уравнение \(AX=B\) для следующих матриц
\begin{enumerate}
	\item \(A=\begin{pmatrix}
		1 & 1 & 0 \\ 1 & 1 & 1 \\ 0 & 1 & 1
	\end{pmatrix}\), \(B=\begin{pmatrix}
		1 & 0 \\ 1 & 1 \\ 0 & 1
	\end{pmatrix}\)
	\item \(A=\begin{pmatrix}
		1 & 1 & -1 \\ 0 & 1 & 1 \\ 0 & 0 & 1
	\end{pmatrix}\), \(B=\begin{pmatrix}
		1 & 0 & -1 \\ 1 & 1 & 0\\ 0 & 1 & -1
	\end{pmatrix}\)
	\item \(A=\begin{pmatrix}
		0 & 1 & 0 \\ -1 & 0 & -1 \\ 0 & 1 & 0
	\end{pmatrix}\), \(B=\begin{pmatrix}
		1 & 2 & 0 & 1 \\ 1 & -1 & 1 & 0\\ 1 & 0 & 1 & 0
	\end{pmatrix}\)
\end{enumerate}
\end{exercise}

\begin{exercise}
Решите матричное уравнение \(XA=B\) для следующих матриц
\begin{enumerate}
	\item \(A=\begin{pmatrix}
		1 & 1 \\ 0 & 1
	\end{pmatrix}\), \(B=\begin{pmatrix}
		1 & 0 \\ 1 & 1
	\end{pmatrix}\)
	\item \(A=\begin{pmatrix}
		1 & 1 \\ 1 & 0
	\end{pmatrix}\), \(B=\begin{pmatrix}
		1 & 2 \\ 1 & 1 \\ 1 & 0
	\end{pmatrix}\)
	\item \(A=\begin{pmatrix}
		1 & 1 & 0 \\ 1 & 1 & 1 \\ 0 & 1 & 1
	\end{pmatrix}\), \(B=\begin{pmatrix}
		1 & 0 & -1 \\ 1 & 1 & 0\\ 0 & 1 & -1
	\end{pmatrix}\)
	\item \(A=\begin{pmatrix}
		0 & 1 & 0 \\ -1 & 0 & -1 \\ 0 & 1 & 0
	\end{pmatrix}\), \(B=\begin{pmatrix}
		1 & 0 & 1 \\ 1 & 1 & 0
	\end{pmatrix}\)
\end{enumerate}
\end{exercise}

\begin{exercise}
Решите матричное уравнение \(A_1XA_2=B\) для следующих матриц
\begin{enumerate}
	\item \(A_1=\begin{pmatrix}
		1 & 1 \\ 0 & 1
	\end{pmatrix}\), \(A_2=\begin{pmatrix}
		0 & 1 \\ 1 & 0
	\end{pmatrix}\), \(B=\begin{pmatrix}
		1 & 0 \\ 1 & 0
	\end{pmatrix}\)
	\item \(A_1=\begin{pmatrix}
		3 & 2 \\ 5 & 3
	\end{pmatrix}\), \(A_2=\begin{pmatrix}
		0 & 1 \\ 1 & 1
	\end{pmatrix}\), \(B=\begin{pmatrix}
		1 & 3 \\ 4 & 2
	\end{pmatrix}\)
	\item \(A_1=\begin{pmatrix}
		0 & 0 & 1 \\ 0 & 1 & -1 \\ 1 & 1 & 0
	\end{pmatrix}\), \(A_2=\begin{pmatrix}
		0 & 1 & 0 \\ 0 & 0 & 1 \\ 1 & 0 & 0
	\end{pmatrix}\), \(B=\begin{pmatrix}
		1 & 0 & -1 \\ 1 & 0 & 1 \\ 0 & 1 & 1
	\end{pmatrix}\)
	\item \(A_1=\begin{pmatrix}
		0 & -1 \\ 1 & 0
	\end{pmatrix}\), \(A_2=\begin{pmatrix}
		0 & 0 & 1 \\ 1 & 0 & 0 \\ 0 & 1 & 0
	\end{pmatrix}\), \(B=\begin{pmatrix}
		1 & 0 & 2 \\ 2 & 2 & 3
	\end{pmatrix}\)
	\item \(A_1=\begin{pmatrix}
		1 & 1 & 1 \\ 0 & 1 & 1 \\ 0 & 0 & 1
	\end{pmatrix}\), \(A_2=\begin{pmatrix}
		0 & 1 \\ 1 & 1
	\end{pmatrix}\), \(B=\begin{pmatrix}
		1 & 1 \\ 1 & 0 \\ 0 & 2
	\end{pmatrix}\)
\end{enumerate}
\end{exercise}

\subsubsection{Определитель}

\begin{exercise}
Вычислите определитель следующих матриц
\begin{align*}
	& \begin{pmatrix}
		2 & 5 \\ 1 & 3
	\end{pmatrix} &
	& \begin{pmatrix}
		3 & 2 \\ 5 & 3
	\end{pmatrix} &
	& \begin{pmatrix}
		2 & 5 \\ 1 & 3
	\end{pmatrix} &
	& \begin{pmatrix}
		2 & 3 \\ 6 & 9
	\end{pmatrix} \\
	& \begin{pmatrix}
		0 & 1 & 0 \\ 0 & 0 & 1 \\ 1 & 0 & 0
	\end{pmatrix}
\end{align*}
\end{exercise}

\begin{exercise}
Вычислите определитель следующих матриц
\begin{align*}
	& \begin{pmatrix}
		0 & 1 & 0 \\ 0 & 0 & 1 \\ 1 & 0 & 0
	\end{pmatrix} &
	& \begin{pmatrix}
		0 & 0 & 1 \\ 0 & 1 & 1 \\ 1 & 1 & 1
	\end{pmatrix} &
	& \begin{pmatrix}
		0 & 1 & 1 \\ 1 & 0 & 1 \\ 1 & 1 & 0
	\end{pmatrix} &
	& \begin{pmatrix}
		0 & 1 & 0 \\ -1 & 0 & -1 \\ 0 & 1 & 0
	\end{pmatrix}
\end{align*}
\end{exercise}


% \appendix

% \section{Приложение}

% \input{appendix.tex}

\end{document}