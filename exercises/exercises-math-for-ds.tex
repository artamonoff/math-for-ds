\documentclass[12pt]{article}

\usepackage[utf8]{inputenc}
\usepackage[T2A]{fontenc}
\usepackage[english, russian]{babel}
\usepackage{amsmath, amsthm, amssymb}
\usepackage{enumerate,hhline, bm}
\usepackage[mathscr]{eucal}
%\usepackage{mathtext}

\usepackage{hyperref}
\hypersetup{unicode=true, final=true, colorlinks=true}


% 
%   Вероятностные определения
%
\DeclareMathOperator{\cov}{cov}
\DeclareMathOperator{\corr}{corr}
\DeclareMathOperator*{\plim}{plim}
\DeclareMathOperator{\Var}{Var}
\DeclareMathOperator{\VVar}{V}
\newcommand{\StdDev}{s.d.}

%
%  Эконометрические
%
\DeclareMathOperator{\const}{const}
\DeclareMathOperator{\error}{error}

%
%  Линейная алгебра
%
\DeclareMathOperator{\rank}{rank}
\DeclareMathOperator{\dimension}{dim}
\DeclareMathOperator{\tr}{tr}
\newcommand{\LinearSpace}{{\mathfrak L}}
\newcommand{\spaceX}{{\mathbb X}}
\newcommand{\spaceY}{{\mathbb Y}}


%
%   Числовые
%
\newcommand{\Complex}{{\mathbb C}}
\newcommand{\N}{\mathbb N}
\newcommand{\Z}{{\mathbb Z}}
\newcommand{\Q}{{\mathbb Q}}
\newcommand{\R}{{\mathbb R}}
\newcommand{\semiaxes}{{\mathbb R_+}}

%
%  Вероятностные
%
\newcommand{\iid}{{i.i.d.}}
\newcommand{\Exp}{{\mathsf E}}
\newcommand{\Gauss}{{\mathscr N}}
\newcommand{\Likelihood}{{\mathcal L}}
\newcommand{\StError}{{s.e.}}
\newcommand{\ConfInterval}{{\mathcal I}}

%
%   Вектора
%
\newcommand{\vconst}{{\mathbf const}}
\newcommand{\vectx}{{\bm x}}
\newcommand{\vecty}{{\bm y}}
\newcommand{\vectz}{{\bm z}}
\newcommand{\vecte}{{\bm e}}
\newcommand{\vectw}{{\bm w}}
\newcommand{\vecth}{{\bm h}}
\newcommand{\vectr}{{\bm r}}
\newcommand{\vectq}{{\bm q}}
\newcommand{\vectf}{{\bm f}}%{\boldsymbol{f}}
\newcommand{\vectu}{{\bm u}}
\newcommand{\vectv}{{\bm v}}
\newcommand{\vectalpha}{{\bm{\alpha}}}
\newcommand{\vectbeta}{{\bm{\beta}}}
\newcommand{\vectgamma}{{\bm{\gamma}}}
\newcommand{\vectdelta}{{\bm{\delta}}}
\newcommand{\vectomega}{{\bm{\omega}}}
\newcommand{\vecttheta}{{\bm{\theta}}}
\newcommand{\vecteta}{{\bm{\eta}}}
\newcommand{\vectpi}{{\bm{\pi}}}
\newcommand{\vectmu}{{\bm{\mu}}}
\newcommand{\vectxi}{{\bm{\xi}}}
\newcommand{\vectX}{{\bm X}}
\newcommand{\vectY}{{\bm Y}}
\newcommand{\vectZ}{{\bm Z}}
\newcommand{\vectones}{{ 1}}

% 
%  Матрицы
%
\newcommand{\Id}{I}
\newcommand{\matrixX}{{\bm X}}
\newcommand{\matrixY}{{\bm Y}}
\newcommand{\matrixU}{{\bm U}}
\newcommand{\matrixV}{{\bm V}}
\newcommand{\matrixR}{{\bm R}}
\newcommand{\matrixZ}{{\bm Z}}
\newcommand{\matrixA}{{\bm A}}
\newcommand{\matrixB}{{\bm B}}
\newcommand{\matrixQ}{{\bm Q}}
\newcommand{\matrixH}{{\bm H}}
\newcommand{\matrixXX}{X}
\newcommand{\matrixGamma}{{\bm{\Gamma}}}
\newcommand{\matrixPi}{{\bm{\Pi}}}

%
% Теоремы, Примеры etc
%
\newtheorem*{teorema}{Теорема}
\newtheorem*{importante}{Важно!}
\newtheorem*{ejemplo}{Пример}
\newtheorem*{definicion}{Определение}
\newtheorem*{proposition}{Предложение}

\theoremstyle{remark}
\newtheorem*{remark}{Замечание}


% \theoremstyle{plain}
% \newtheorem*{trm}{Теорема}
% \newtheorem*{mprtnt}{Важно!}
% \newtheorem*{xmpl}{Пример}
% \newtheorem*{dfntn}{Определение}

% \theoremstyle{remark}
% \newtheorem*{rmrk}{Замечание}

\theoremstyle{remark}
\newtheorem{exercise}{}[subsection]
\renewcommand{\theexercise}{\textbf{\textnumero \arabic{exercise}}}

%\DeclareMathOperator{\cov}{cov}
%\DeclareMathOperator{\corr}{corr}
%\DeclareMathOperator{\Var}{Var}

%\topmargin=-2cm%-1.5cm

%\addtolength{\textheight}{3cm}

%\oddsidemargin=-0.1cm

%\addtolength{\textwidth}{1.8cm}


\title{Задачи по Математическим основам анализа данных}
\author{Артамонов Н.В.}
%\date{весна 2014}

%\title{Задачи для подготовки к экзамену по курсу 
%<<Методы оптимальных решений>>}\author{\copyright Артамонов Н.В., кафедра ЭММАЭ}

\begin{document}

\maketitle

%\markright{}
\tableofcontents

\section{Работа с массивами (матричный анализ)}

% !TEX root = exercises-math-for-ds.tex

\subsection{Операции с матрицами}

\begin{exercise}
Рассмотрим матрицы
\begin{align*}
	A&=\begin{pmatrix}
		-1 & 2 & 0 \\ 0 & 2 & 3 \\ 1 & -1 & 0 \\ 2 & -2 & 0
	\end{pmatrix} &
	B&=\begin{pmatrix}
		0 & -1 & 1 \\ 0 & 1 & 1 \\ 2 & 0 & -1 \\ 1 & 1 & 2
	\end{pmatrix} &
	C&=\begin{pmatrix}
		2 & 0 & -1 \\ -1 & -2 & 2 \\ 1 & -1& 2 \\ 0 & 3 & -1
	\end{pmatrix}
\end{align*}
Вычислите
\begin{align*}
	& 2A+B & &A-2C & &4B-A-C & &C-2A+4B
\end{align*}
\end{exercise}

\begin{exercise}
Рассмотрим матрицы
\begin{align*}
	A&=\begin{pmatrix}
		2 & 1 & 5 \\ 3 & 4 & 3 \\ 1 & 2 & 0 \\ 2 & 3 & 1 \\ 1 & 1 & 0
	\end{pmatrix} &
	B&=\begin{pmatrix}
		2 & 1 & 0 \\ 2 & 5 & 2 \\ 4 & 3 & 2 \\ 3 & 4 & 1 \\ 1 & 3 & 2
	\end{pmatrix} &
	C&=\begin{pmatrix}
		5 & 2 & 3 \\ 2 & 3 & 0 \\ 2 & 1 & 0 \\ 1 & 0 & 1 \\ 2 & 2 & 3
	\end{pmatrix}
\end{align*}
Вычислите
\begin{align*}
	& A+3B & &3B-2C & &2B-C+3A & &2C+3A-5B
\end{align*}
\end{exercise}

\begin{exercise}
Рассмотрим матрицы
\begin{align*}
	A&=\begin{pmatrix}
		-1 & 2 & 2 & 1 & 0 \\ 1 & 0 & -2 & 1 & 0
	\end{pmatrix} \\
	B&=\begin{pmatrix}
		0 & 0 & 1 & 3 & 2 \\ -1 & 0 & 2 & 1 & -3
	\end{pmatrix} \\
	C&=\begin{pmatrix}
		1 & 2 & 0 & 1 & 0 \\ 0 & 1 & 1 & -1 & -2
	\end{pmatrix}
\end{align*}
Вычислите
\begin{align*}
	& 3A-B & &2A-C & &2B-C+3A & &B-2A+C
\end{align*}
\end{exercise}

\section{Элементы анализа}

% !TEX root = exercises-math-for-ds.tex

\subsection{Функции одной переменной}

\begin{exercise}
Вычислите первую производную функций
\begin{align*}
	f(x)&=x\cos(x) & f(x)&=x\sin(x) & f(x)&=x^2\sin(x) & f(x)&=x^2\cos(x) \\
	f(x)&=\cos^2(x) & f(x)&=\sin^2(x) & f(x)&=x\cos^2(x) & f(x)&=x\sin^2(x) \\
	f(x)&=\frac{\sin(x)}{x} & f(x)&=\frac{\cos(x)}{x} & f(x)&=\frac{\cos^2{x}}{x} & f(x)&=\frac{\sin{x}}{x^2} \\
	f(x)&=x\ln x & f(x)&=x^2\ln x & f(x)&=x\ln^2(x) & f(x)&=\frac{\ln x}{x} \\
	f(x)&=x\exp(x) & f(x)&=\exp(x^2) & f(x)&=x\exp(-x) & f(x)&=x\exp(-x^2) \\
\end{align*}
\end{exercise}

\begin{exercise}
Вычислите значение первой производной функции
\begin{enumerate}
	\item \(f(x)=x\cos(x)\) в точках \(x=0, \pi/2, \pi\)
	\item \(f(x)=x^2\sin(x)\) в точках \(x=1, \pi/2, \pi\)
	\item \(f(x)=x^3\ln x\) в точках \(x=1, 2, 3\)
	\item \(f(x)=x\exp(x^2)\) в точках \(x=1, 2, 3\)
\end{enumerate}
\end{exercise}

\begin{exercise}
Вычислите вторую производную функций
\begin{align*}
	f(x)&=x\cos(x) & f(x)&=x\sin(x) & f(x)&=\cos^2(x) & f(x)&=\sin^2(x) \\
	f(x)&=x\ln x & f(x)&=x\exp(x) & f(x)&=x^2\exp(-x) & f(x)&=\exp(-x^2)
\end{align*}
\end{exercise}

\begin{exercise}
Найдите (численно) локальные экстремумы функции 
\begin{align*}
	f(x)&=10+3x-x^2 & f(x)&=2x^2+4x-5 \\
	f(x)&=x^3-4x^2+3x-10 & f(x)&=6+3x-5x^2-x^3 \\
	f(x)&=x\exp(x) & f(x)&=x^2\exp(x) \\
	f(x)&=x^3\exp(-x) & f(x)&=x\exp(-x^2) 
\end{align*}
\end{exercise}

\subsection{Функции многих переменных}

\begin{exercise}
Вычислите градиент следующих функции
\begin{align*}
	f&=xy & f&=x^2y^2 &  f&=x^2y-xy^2 & f&=x^2-xy+y^2 \\
	f&=\exp(xy) & f&=\exp(x+y) & f&=\ln(x+y) & f&=\exp(x^2y)
\end{align*}
\end{exercise}

\begin{exercise}
Найдите значение градиента функции
\begin{enumerate}
	\item \(f=xy^2\) в точке \((1,2)\)
	\item \(f=x^2y+xy^2\) в точке \((2,-1)\)
	\item \(f=x^2+xy+y^2\) в точке \((-1,2)\)
	\item \(f=\ln(x^2+y^2)\) в точке \((2,3)\)
	\item \(f=\exp(x^2+y^2)\) в точке \((-2,1)\)
\end{enumerate}
\end{exercise}

\begin{exercise}
Найдите локальные экстремумы функций
\begin{align*}
	f(x,y) &= 10-6x-4y+2x^2+y^2-2xy \\
	f(x,y) &= 8+8x+4y-5x^2-2y^2+6xy \\
	f(x,y) &= 5+2x+6y+5x^2+3y^2+8xy
\end{align*}
% Нарисуйте графики функций (MS Excel, Python etc)
\end{exercise}

\begin{exercise}
Найдите локальные экстремумы функций
\begin{align*}
	f(x,y,z) &= 6+4x+2y+6z+2x^2+2y^2+z^2+2xy+2yz \\
	f(x,y,z) &= 3+4x+8y+4z-3x^2-2y^2-4z^2+2xy+2xz+4yz\\
	f(x,y,z) &= 8+2x+4y+2z+2x^2+y^2+3z^2+2xy+4xz+4yz
\end{align*}
\end{exercise}
	
\begin{exercise}
Найдите локальные экстремумы функций
\begin{align*}
	f(x,y) &= 5+x^3-y^3+3xy \\
	f(x,y) &= 3x^2y+y^3-3x^2-3y^2+2 \\
	f(x,y) &= x^3+x^2y-2y^3+6y
\end{align*}
\end{exercise}

\begin{exercise}
Найдите локальные экстремумы функций
\begin{align*}
	f(x,y) &= 6\ln x+8\ln y-3x-2y \\
	f(x,y) &= 4\ln x+6\ln y+2x-3xy \\
	f(x,y) &= 5\ln x+4\ln y-x-4xy
\end{align*}
\end{exercise}

\section{Элементы теории вероятностей}

% !TEX root = exercises-math-for-ds.tex

\subsection{Дискретные распределения}

\begin{exercise}
Симметричную монетку бросают 5 раз. Случайная величина \(X\) -- число выпадений Орла.
Вычислите
\begin{enumerate}
	\item \(\Exp(X), Var(X), \sigma(X)\)
	\item вероятности
	\begin{align*}
		&\Prob(X=0) & &\Prob(X=2) & &\Prob(X\leq3) & &\Prob(X>2)
	\end{align*}
	\item моду распределения
\end{enumerate}
Визуализируйте \(f(k)=\Prob(X=k)\)
\end{exercise}

\begin{exercise}
Симметричную монетку бросают 8 раз. Случайная величина \(X\) -- число выпадений Орла.
Вычислите
\begin{enumerate}
	\item \(\Exp(X), Var(X), \sigma(X)\)
	\item вероятности
	\begin{align*}
		&\Prob(X=3) & &\Prob(X=2) & &\Prob(X\leq3) & &\Prob(X>2)
	\end{align*}
	\item моду распределения
\end{enumerate}
Визуализируйте \(f(k)=\Prob(X=k)\)
\end{exercise}

\begin{exercise}
Несимметричную монетку бросают 7 раз. Вероятность выпадения Герба в одном бросании равна 0.4. 
Случайная величина \(X\) -- число выпадений Герба. Вычислите
\begin{enumerate}
	\item \(\Exp(X), Var(X), \sigma(X)\)
	\item вероятности
	\begin{align*}
		&\Prob(X=3) & &\Prob(X=6) & &\Prob(X\leq4) & &\Prob(X>3)
	\end{align*}
	\item моду распределения
\end{enumerate}
Визуализируйте \(f(k)=\Prob(X=k)\)
\end{exercise}

\begin{exercise}
Вероятность выпуска прибора, удовлетворяющего требованиям качества, равна 0.9.
В контрольной партии 5 приборов. Случайная величина \(X\) -- число приборов в партии,
удовлетворяющих требованиям качества. Вычислите вероятность, что в партии
будет не менее 4 приборов надлежащего качества.
\end{exercise}

\begin{exercise}
Вероятность попадания в цель при одном выстреле равна 0.7. Случайная
величина \(X\) -- число попаданий в цель при четырёх выстрелах.
Вычислите вероятность того, что будет не менее двух попаданий в цель.
\end{exercise}

\begin{exercise}
Автомобиль должен проехать по улице, на которой установлено пять независимо работающих светофоров. 
Каждый светофор с интервалом в 3 мин подает красный и зеленый сигналы. 
Случайная величина \(X\) -- число остановок на этой улице.
Вычислите вероятность того, что будет не более трёх остановок.
\end{exercise}

\begin{exercise}
В урне 8 шаров, 4 из них белые, а остальные чёрные. Случайным образом выбрали 4 шара. 
Случайная величина \(X\) -- число белых шаров среди выбранных. 
Вычислите
\begin{enumerate}
	\item \(\Exp(X), Var(X), \sigma(X)\)
	\item вероятности
	\begin{align*}
		&\Prob(X=2) & &\Prob(X=3) & &\Prob(X\leq2) & &\Prob(X>1)
	\end{align*}
	\item моду распределения
\end{enumerate}
Визуализируйте \(f(k)=\Prob(X=k)\)
\end{exercise}

\begin{exercise}
В группе из шести человек два отличника. Случайным образом выбрали двух человек. 
Случайная величина \(X\) -- число отличников среди выбранных. 
Вычислите
\begin{enumerate}
	\item \(\Exp(X), Var(X), \sigma(X)\)
	\item вероятности
	\begin{align*}
		&\Prob(X=1) & &\Prob(X=2) & &\Prob(X\leq2) & &\Prob(X>0)
	\end{align*}
	\item моду распределения
\end{enumerate}
Визуализируйте \(f(k)=\Prob(X=k)\)
\end{exercise}

\begin{exercise}
Партия из 15 изделий содержит 5 бракованных. Из партии случайным образом взято 4 изделия. 
Пусть \(X\) -- число бракованных изделий среди трех взятых.
Вычислите
\begin{enumerate}
	\item \(\Exp(X), Var(X), \sigma(X)\)
	\item вероятности
	\begin{align*}
		&\Prob(X=1) & &\Prob(X=4) & &\Prob(X\leq3) & &\Prob(X>2)
	\end{align*}
	\item моду распределения
\end{enumerate}
Визуализируйте \(f(k)=\Prob(X=k)\)
\end{exercise}

\subsection{Непрерывные распределения}

\begin{exercise}
Для распределения \(\Gauss(0,1)\) вычислите
\begin{align*}
	&\phi(1) & &\phi(2) & &\phi(-0.5) & &\phi(-1.5) & &\Phi(1) & &\Phi(2) & &\Phi(-1) & &\Phi(-2)
\end{align*}
\end{exercise}

\begin{exercise}
Для распределения \(\Gauss(1,0.5^2)\) вычислите значение функции распределения и плотности в точках
\[
	x\in\{-1.5, -1, -0.5, 0, 0.5, 1, 1.5, 2, 2.5\}
\]
\end{exercise}

\begin{exercise}
Пусть \(X\sim\Gauss(0,1)\). Вычислите следующие вероятности
\begin{align*}
	&\Prob(X\leq1) & &\Prob(X>-0.5) & 
	&\Prob(-1\leq X\leq0.5) & &\Prob(0<X<2)
\end{align*}
\end{exercise}

\begin{exercise}
Пусть \(X\sim\Gauss(1,1.5^2)\). Вычислите следующие вероятности
\begin{align*}
	&\Prob(X\leq2) & &\Prob(X>0.5) & 
	&\Prob(-0.5\leq X\leq1.5) & &\Prob(0<X<3)
\end{align*}
\end{exercise}

\begin{exercise}
Пусть \(X\sim\Gauss(0,1)\). Найдите \(a,b,c\) т.ч.
\begin{align*}
	&\Prob(X\leq a)=0.6 & &\Prob(X\leq b)=0.8 & 
	&\Prob(X\leq c)=0.9
\end{align*}
\end{exercise}

\begin{exercise}
Пусть \(X\sim\Gauss(1,0.5^2)\). Найдите \(a,b,c\) т.ч.
\begin{align*}
	&\Prob(X\leq a)=0.7 & &\Prob(X\leq b)=0.85 & 
	&\Prob(X\leq c)=0.95
\end{align*}
\end{exercise}

\begin{exercise}
Для распределения \(U[1, 4]\) 
вычислите значение функции распределения и плотности в точках
\[
	x\in\{0, 1.5, 2, 2.5, 3, 3.5, 4, 5\}
\]
\end{exercise}

\begin{exercise}
Пусть \(X\sim U[-1, 5]\). Вычислите следующие вероятности
\begin{align*}
	&\Prob(X\leq0) & &\Prob(X>2) & 
	&\Prob(-0.5\leq X\leq3.5) & &\Prob(0<X<4)
\end{align*}
\end{exercise}

\subsection{Критические значения}

\textbf{Замечание}: все вычисления необходимо сделать в MS Excel/Python

\begin{exercise}
Для уровней значимости: 1\%, 5\%, 10\% вычислите (двусторонние) 
критические значения распределения \(\Gauss(0,1)\)
\end{exercise}

\begin{exercise}
Для уровней значимости: 1\%, 5\%, 10\% вычислите (двусторонние) 
критические значения следующих распределений
\begin{align*}
	&t_{10} & &t_{100} & &t_{250} & &t_{500}
\end{align*}
\end{exercise}

\begin{exercise}
Для уровней значимости: 1\%, 5\%, 10\% вычислите
критические значения следующих распределений
\begin{align*}
	&\chi^2_{2} & &\chi^2_{5} & &\chi^2_{10} & &\chi^2_{20}
\end{align*}
\end{exercise}

\begin{exercise}
Для уровней значимости: 1\%, 5\%, 10\% вычислите
критические значения следующих распределений
\begin{align*}
	&F_{2,100} & &F_{5, 300} & &F_{10, 1000} & &F_{20, 1500}
\end{align*}
\end{exercise}


% \appendix

% \section{Приложение}

% \input{appendix.tex}

\end{document}