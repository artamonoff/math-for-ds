% !TEX root = exercises-math-for-ds.tex

\subsection{Дискретные распределения}

\begin{exercise}
Симметричную монетку бросают 5 раз. Случайная величина \(X\) -- число выпадений Орла.
Вычислите
\begin{enumerate}
	\item \(\Exp(X), Var(X), \sigma(X)\)
	\item вероятности
	\begin{align*}
		&\Prob(X=0) & &\Prob(X=2) & &\Prob(X\leq3) & &\Prob(X>2)
	\end{align*}
	\item моду распределения
\end{enumerate}
Визуализируйте \(f(k)=\Prob(X=k)\)
\end{exercise}

\begin{exercise}
Симметричную монетку бросают 8 раз. Случайная величина \(X\) -- число выпадений Орла.
Вычислите
\begin{enumerate}
	\item \(\Exp(X), Var(X), \sigma(X)\)
	\item вероятности
	\begin{align*}
		&\Prob(X=3) & &\Prob(X=2) & &\Prob(X\leq3) & &\Prob(X>2)
	\end{align*}
	\item моду распределения
\end{enumerate}
Визуализируйте \(f(k)=\Prob(X=k)\)
\end{exercise}

\begin{exercise}
Несимметричную монетку бросают 7 раз. Вероятность выпадения Герба в одном бросании равна 0.4. 
Случайная величина \(X\) -- число выпадений Герба. Вычислите
\begin{enumerate}
	\item \(\Exp(X), Var(X), \sigma(X)\)
	\item вероятности
	\begin{align*}
		&\Prob(X=3) & &\Prob(X=6) & &\Prob(X\leq4) & &\Prob(X>3)
	\end{align*}
	\item моду распределения
\end{enumerate}
Визуализируйте \(f(k)=\Prob(X=k)\)
\end{exercise}

\begin{exercise}
Вероятность выпуска прибора, удовлетворяющего требованиям качества, равна 0.9.
В контрольной партии 5 приборов. Случайная величина \(X\) -- число приборов в партии,
удовлетворяющих требованиям качества. Вычислите вероятность, что в партии
будет не менее 4 приборов надлежащего качества.
\end{exercise}

\begin{exercise}
Вероятность попадания в цель при одном выстреле равна 0.7. Случайная
величина \(X\) -- число попаданий в цель при четырёх выстрелах.
Вычислите вероятность того, что будет не менее двух попаданий в цель.
\end{exercise}

\begin{exercise}
Автомобиль должен проехать по улице, на которой установлено пять независимо работающих светофоров. 
Каждый светофор с интервалом в 3 мин подает красный и зеленый сигналы. 
Случайная величина \(X\) -- число остановок на этой улице.
Вычислите вероятность того, что будет не более трёх остановок.
\end{exercise}

\begin{exercise}
В урне 8 шаров, 4 из них белые, а остальные чёрные. Случайным образом выбрали 4 шара. 
Случайная величина \(X\) -- число белых шаров среди выбранных. 
Вычислите
\begin{enumerate}
	\item \(\Exp(X), Var(X), \sigma(X)\)
	\item вероятности
	\begin{align*}
		&\Prob(X=2) & &\Prob(X=3) & &\Prob(X\leq2) & &\Prob(X>1)
	\end{align*}
	\item моду распределения
\end{enumerate}
Визуализируйте \(f(k)=\Prob(X=k)\)
\end{exercise}

\begin{exercise}
В группе из шести человек два отличника. Случайным образом выбрали двух человек. 
Случайная величина \(X\) -- число отличников среди выбранных. 
Вычислите
\begin{enumerate}
	\item \(\Exp(X), Var(X), \sigma(X)\)
	\item вероятности
	\begin{align*}
		&\Prob(X=1) & &\Prob(X=2) & &\Prob(X\leq2) & &\Prob(X>0)
	\end{align*}
	\item моду распределения
\end{enumerate}
Визуализируйте \(f(k)=\Prob(X=k)\)
\end{exercise}

\begin{exercise}
Партия из 15 изделий содержит 5 бракованных. Из партии случайным образом взято 4 изделия. 
Пусть \(X\) -- число бракованных изделий среди трех взятых.
Вычислите
\begin{enumerate}
	\item \(\Exp(X), Var(X), \sigma(X)\)
	\item вероятности
	\begin{align*}
		&\Prob(X=1) & &\Prob(X=4) & &\Prob(X\leq3) & &\Prob(X>2)
	\end{align*}
	\item моду распределения
\end{enumerate}
Визуализируйте \(f(k)=\Prob(X=k)\)
\end{exercise}

\subsection{Непрерывные распределения}

\begin{exercise}
Для распределения \(\Gauss(0,1)\) вычислите
\begin{align*}
	&\phi(1) & &\phi(2) & &\phi(-0.5) & &\phi(-1.5) & &\Phi(1) & &\Phi(2) & &\Phi(-1) & &\Phi(-2)
\end{align*}
\end{exercise}

\begin{exercise}
Для распределения \(\Gauss(1,0.5^2)\) вычислите значение функции распределения и плотности в точках
\[
	x\in\{-1.5, -1, -0.5, 0, 0.5, 1, 1.5, 2, 2.5\}
\]
\end{exercise}

\begin{exercise}
Пусть \(X\sim\Gauss(0,1)\). Вычислите следующие вероятности
\begin{align*}
	&\Prob(X\leq1) & &\Prob(X>-0.5) & 
	&\Prob(-1\leq X\leq0.5) & &\Prob(0<X<2)
\end{align*}
\end{exercise}

\begin{exercise}
Пусть \(X\sim\Gauss(1,1.5^2)\). Вычислите следующие вероятности
\begin{align*}
	&\Prob(X\leq2) & &\Prob(X>0.5) & 
	&\Prob(-0.5\leq X\leq1.5) & &\Prob(0<X<3)
\end{align*}
\end{exercise}

\begin{exercise}
Пусть \(X\sim\Gauss(0,1)\). Найдите \(a,b,c\) т.ч.
\begin{align*}
	&\Prob(X\leq a)=0.6 & &\Prob(X\leq b)=0.8 & 
	&\Prob(X\leq c)=0.9
\end{align*}
\end{exercise}

\begin{exercise}
Пусть \(X\sim\Gauss(1,0.5^2)\). Найдите \(a,b,c\) т.ч.
\begin{align*}
	&\Prob(X\leq a)=0.7 & &\Prob(X\leq b)=0.85 & 
	&\Prob(X\leq c)=0.95
\end{align*}
\end{exercise}

\begin{exercise}
Для распределения \(U[1, 4]\) 
вычислите значение функции распределения и плотности в точках
\[
	x\in\{0, 1.5, 2, 2.5, 3, 3.5, 4, 5\}
\]
\end{exercise}

\begin{exercise}
Пусть \(X\sim U[-1, 5]\). Вычислите следующие вероятности
\begin{align*}
	&\Prob(X\leq0) & &\Prob(X>2) & 
	&\Prob(-0.5\leq X\leq3.5) & &\Prob(0<X<4)
\end{align*}
\end{exercise}

\subsection{Критические значения}

\textbf{Замечание}: все вычисления необходимо сделать в MS Excel/Python

\begin{exercise}
Для уровней значимости: 1\%, 5\%, 10\% вычислите (двусторонние) 
критические значения распределения \(\Gauss(0,1)\)
\end{exercise}

\begin{exercise}
Для уровней значимости: 1\%, 5\%, 10\% вычислите (двусторонние) 
критические значения следующих распределений
\begin{align*}
	&t_{10} & &t_{100} & &t_{250} & &t_{500}
\end{align*}
\end{exercise}

\begin{exercise}
Для уровней значимости: 1\%, 5\%, 10\% вычислите
критические значения следующих распределений
\begin{align*}
	&\chi^2_{2} & &\chi^2_{5} & &\chi^2_{10} & &\chi^2_{20}
\end{align*}
\end{exercise}

\begin{exercise}
Для уровней значимости: 1\%, 5\%, 10\% вычислите
критические значения следующих распределений
\begin{align*}
	&F_{2,100} & &F_{5, 300} & &F_{10, 1000} & &F_{20, 1500}
\end{align*}
\end{exercise}