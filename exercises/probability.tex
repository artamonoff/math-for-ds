% !TEX root = exercises-math-for-ds.tex

\subsection{Дискретные распределения}

\begin{exercise}
Симметричную монетку бросают 5 раз. Случайная величина \(X\) -- число выпадений Орла.
Вычислите
\begin{enumerate}
	\item \(\Exp(X), Var(X), \sigma(X)\)
	\item вероятности
	\begin{align*}
		&\Prob(X=0) & &\Prob(X=2) & &\Prob(X\leq3) & &\Prob(X>2)
	\end{align*}
	\item моду распределения
\end{enumerate}
Визуализируйте \(f(k)=\Prob(X=k)\)
\end{exercise}

\begin{exercise}
Симметричную монетку бросают 8 раз. Случайная величина \(X\) -- число выпадений Орла.
Вычислите
\begin{enumerate}
	\item \(\Exp(X), Var(X), \sigma(X)\)
	\item вероятности
	\begin{align*}
		&\Prob(X=3) & &\Prob(X=2) & &\Prob(X\leq3) & &\Prob(X>2)
	\end{align*}
	\item моду распределения
\end{enumerate}
Визуализируйте \(f(k)=\Prob(X=k)\)
\end{exercise}

\begin{exercise}
Несимметричную монетку бросают 7 раз. Вероятность выпадения Герба в одном бросании равна 0.4. 
Случайная величина \(X\) -- число выпадений Герба. Вычислите
\begin{enumerate}
	\item \(\Exp(X), Var(X), \sigma(X)\)
	\item вероятности
	\begin{align*}
		&\Prob(X=3) & &\Prob(X=6) & &\Prob(X\leq4) & &\Prob(X>3)
	\end{align*}
	\item моду распределения
\end{enumerate}
Визуализируйте \(f(k)=\Prob(X=k)\)
\end{exercise}

\begin{exercise}
Вероятность выпуска прибора, удовлетворяющего требованиям качества, равна 0.9.
В контрольной партии 5 приборов. Случайная величина \(X\) -- число приборов в партии,
удовлетворяющих требованиям качества. Вычислите вероятность, что в партии
будет не менее 4 приборов надлежащего качества.
\end{exercise}

\begin{exercise}
Вероятность попадания в цель при одном выстреле равна 0.7. Случайная
величина \(X\) -- число попаданий в цель при четырёх выстрелах.
Вычислите вероятность того, что будет не менее двух попаданий в цель.
\end{exercise}

\begin{exercise}
Автомобиль должен проехать по улице, на которой установлено пять независимо работающих светофоров. 
Каждый светофор с интервалом в 3 мин подает красный и зеленый сигналы. 
Случайная величина \(X\) -- число остановок на этой улице.
Вычислите вероятность того, что будет не более трёх остановок.
\end{exercise}

\begin{exercise}
В урне 8 шаров, 4 из них белые, а остальные чёрные. Случайным образом выбрали 4 шара. 
Случайная величина \(X\) -- число белых шаров среди выбранных. 
Вычислите
\begin{enumerate}
	\item \(\Exp(X), Var(X), \sigma(X)\)
	\item вероятности
	\begin{align*}
		&\Prob(X=2) & &\Prob(X=3) & &\Prob(X\leq2) & &\Prob(X>1)
	\end{align*}
	\item моду распределения
\end{enumerate}
Визуализируйте \(f(k)=\Prob(X=k)\)
\end{exercise}

\begin{exercise}
В группе из шести человек два отличника. Случайным образом выбрали двух человек. 
Случайная величина \(X\) -- число отличников среди выбранных. 
Вычислите
\begin{enumerate}
	\item \(\Exp(X), Var(X), \sigma(X)\)
	\item вероятности
	\begin{align*}
		&\Prob(X=1) & &\Prob(X=2) & &\Prob(X\leq2) & &\Prob(X>0)
	\end{align*}
	\item моду распределения
\end{enumerate}
Визуализируйте \(f(k)=\Prob(X=k)\)
\end{exercise}

\begin{exercise}
Партия из 15 изделий содержит 5 бракованных. Из партии случайным образом взято 4 изделия. 
Пусть \(X\) -- число бракованных изделий среди трех взятых.
Вычислите
\begin{enumerate}
	\item \(\Exp(X), Var(X), \sigma(X)\)
	\item вероятности
	\begin{align*}
		&\Prob(X=1) & &\Prob(X=4) & &\Prob(X\leq3) & &\Prob(X>2)
	\end{align*}
	\item моду распределения
\end{enumerate}
Визуализируйте \(f(k)=\Prob(X=k)\)
\end{exercise}